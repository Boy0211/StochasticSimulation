%-------------------------------------------------------------------------------
% LATEX TEMPLATE ARTIKEL
%-------------------------------------------------------------------------------
% Dit template is voor gebruik door studenten van de de bacheloropleiding 
% Informatica van de Universiteit van Amsterdam.
% Voor informatie over schrijfvaardigheden, zie 
%                               https://practicumav.nl/schrijven/index.html
%
%-------------------------------------------------------------------------------
%	PACKAGES EN DOCUMENT CONFIGURATIE
%-------------------------------------------------------------------------------

\documentclass{uva-inf-article}
\usepackage[dutch]{babel}

\usepackage[style=authoryear-comp]{biblatex}
\addbibresource{references.bib}

%-------------------------------------------------------------------------------
%	GEGEVENS VOOR IN DE TITEL, HEADER EN FOOTER
%-------------------------------------------------------------------------------

% Geef je artikel een logische titel die de inhoud dekt.
\title{Titel van het document}

% Vul de naam van de opdracht in zoals gegeven door de docent en het type 
% opdracht, bijvoorbeeld 'technisch rapport' of 'essay'.
\assignment{Naam van de opdracht}
\assignmenttype{Type opdracht}

% Vul de volledige namen van alle auteurs in en de corresponderende UvAnetID's.
\authors{Auteur 1; Auteur 2}
\uvanetids{UvAnetID student 1; UvAnetID student 2}

% Vul de naam van je tutor, begeleider (mentor), of docent / vakcoördinator in.
% Vermeld in ieder geval de naam van diegene die het artikel nakijkt!
\tutor{Naam van de tutor}
\mentor{}
\docent{}

% Vul hier de naam van je tutorgroep, werkgroep, of practicumgroep in.
\group{Naam van de groep}

% Vul de naam van de cursus in en de cursuscode, te vinden op o.a. DataNose.
\course{Naam van de cursus}
\courseid{}

% Dit is de datum die op het document komt te staan. Standaard is dat vandaag.
\date{\today}

%-------------------------------------------------------------------------------
%	VOORPAGINA 
%-------------------------------------------------------------------------------

\begin{document}
\maketitle

%-------------------------------------------------------------------------------
%	INHOUDSOPGAVE EN ABSTRACT
%-------------------------------------------------------------------------------
% Niet toevoegen bij een kort artikel, zeg minder dan 10 pagina's!

%TC:ignore
%\tableofcontents
%\begin{abstract}
%\end{abstract}
%TC:endignore

%-------------------------------------------------------------------------------
%	INHOUD
%-------------------------------------------------------------------------------
% Hanteer bij benadering IMRAD: Introduction, Method, Results, Discussion.

\section{Introductie}
\lipsum[1][1-3] 
Deze alinea eindigt met een referentie \parencite{vanwijk21}.
\subsectionauthor[Voornaam Achternaam]{Paragraaf met auteur}
\lipsum[2-3]

%-------------------------------------------------------------------------------
%	REFERENTIES
%-------------------------------------------------------------------------------

\printbibliography

%-------------------------------------------------------------------------------
%	BIJLAGEN 
%-------------------------------------------------------------------------------

%TC:ignore
\appendix 
\section{Bijlage {\LaTeX} code}
Bijgevoegd zijn de \textattachfile{main.tex}{code} en 
\textattachfile{references.bib}{bibliografie}.
%TC:endignore

%-------------------------------------------------------------------------------
\end{document}